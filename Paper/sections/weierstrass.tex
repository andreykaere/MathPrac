\begin{center}
    {\normalfont\fontsize{13}{13}\textbf{Приведение кубики к нормальной форме Вейерштрасса}}
\end{center}

Для нашей кубики 
\[
F(x, y, z) = a_{30} x^3 + a_{03} y^3 + a_{00} z^3 + a_{01} y z^2 + a_{10} x z^2 +
a_{11} x y z + a_{21} x^2 y + a_{12} x y^2 + a_{20} x^2 z + a_{02} y^2 z
,\] 
то есть \(a_{ij} \in \Z\) -- коэффициент при мономе \(x^{i} y^{j} z^{3 - i -
j}\). 



\begin{itemize}[leftmargin=0.6cm]
    
    \item Шаг 0. Нахождение рациональных точек перегиба. 

    Для того, чтобы сделать первый шаг в алгоритме приведения неособой
    кубики к нормальной форме Вейерштрасса целочисленным проективным
    преобразованием нам сначала понадобиться найти рациональную точку
    перегиба. Для этого, мы рассмотрим гессиан нашей кубики \(H(F) := \det \left(
    \frac{\partial F}{\partial x^{i} \partial x^{j}}\right)_{ij}\), где \(x =
    x^{1}, y = x^{2}, z = x^{3}\). Далее, под проективными
    преобразованиями мы будем понимать только целочисленные проективные
    преобразования.

    \begin{theoremf}
        Точки перегиба неособой кубики в \(\C \mathrm{P}^2\) -- в точности
        точки пересечения кубики с её гессианом.
    \end{theoremf}

    Доказательство можно посмотреть в ~\cite{pra_sol}.

    Итак, для нахождения рациональных точек перегиба, нам достаточно найти
    все общие рациональные корни \(F\) и \(H(F)\). Для этого, мы
    рассмотрим \(F\) и \(H(F)\) как многочлены от \(z\):
    \begin{align*}
        F &= b_0(x, y) z^3 + b_1(x, y) z^2 + b_2(x, y) z + b_3(x, y) \\
        H(F) &= c_0(x, y) z^3 + c_1(x, y) z^2 + c_2(x, y) z + c_3(x, y) 
    ,\end{align*}
    где \(b_k(x, y)\) и \(c_k(x, y)\) -- однородные многочлены от \(x, y\) 
    степени \(k\). 

    Если \(c_0(x, y) \cdot b_0(x, y) = 0\), то есть если
    \(F\) или \(H(F)\) проходит через точку \((0 : 0 : 1)\), то мы можем
    сделать проективное преобразование так, что ни одна из кубик не будет
    через неё проходить. Сделать это можно, например, так: найти точку,
    которая не лежит ни на одной из наших кубик и \textquote{обменять} её
    с \((0 : 0 : 1)\). Итак, будем искать точку в виде \((0 : i : 1), \; i
    = 1, 2, \ldots\).
    Так как кубика неособая, то она не распадающаяся (то есть не содержит
    прямой) и различные кубики могут пересекаться не более, чем в \(9\) 
    точках, то процесс перебора закончится не позднее, чем через \(10\) 
    шагов. (Условие \(c_0 b_0 \ne 0\) нужно, чтобы ... ?)

    Далее считаем, что \(c_0(x, y) \cdot b_0(x, y) \ne 0\). Рассмотрим
    \(R(F, H(F))\) -- результант \(F\) и \(H(F)\) по переменной \(z\) (При
    фиксированных \(x, y\) -- это обычный результант двух многочленов).
    Непосредственной проверкой проверяется, что \(R(F, H(F))\) --
    либо однородный многочлен от \(x, y\) степени \(9\), либо
    тождественный ноль. Перебирая все рациональные корни этого многочлена
    -- получаем все потенциальные рациональные точки перегиба. Подставляя
    их в исходное уравнение \(F\), смотрим, есть ли точка с такими
    рациональными координатами по \(x, y\) с рациональной координатой и по
    \(z\). Таким образом, мы находим все рациональные точки перегиба на
    нашей кубике. 
    
    <надо подправить потом>

    Данный алгоритм реализован в файле \textsf{InflectionPoints.py} в функции
    \textsf{find\_inflection\_points} с помощью вспомогательных
    функций \textsf{get\_hessian} и \textsf{intersection\_points}. Далее в функции \textsf{find\_non\_singular\_inflection\_point} мы отбираем
    только неособые точки перегиба.
    

    \item Шаг 1. Выберем какую-то рациональную точку перегиба \(P\).
    Переведём точку \(P\) в точку \((0 : 1 : 0)\) некоторым проективным
    преобразованием. Например, подойдёт такое:  ...
    

    После проективного преобразования, наша кубика имеет уравнение
    \(\widetilde{F}(\widetilde{x}, \widetilde{y}, \widetilde{z})\), в
    котором нет \(\widetilde{y}\). 

    Этот шаг реализован в файле \textsf{WeierstrassForm.py} в функции
    \textsf{weierstrass\_form\_step1}.


    \item Шаг 2. Теперь мы хотим сделать преобразование,
    сохраняющее точку \(O = (0 : 1 : 0)\), так, чтобы
    прямая \(z = 0\) была касательной к кубике \(\widetilde{F}\) в точке
    \(O\), то есть сделать преобразование \((\widetilde{x} :
    \widetilde{y} : \widetilde{z}) \to (x' : y' : z')\), такое что
    для \(F'(x', y', z')\) будет выполнено: 
     \[
    \frac{\partial F'}{\partial x'} (O) = \frac{\partial F'}{\partial y'}
    (O) = 0, \quad \frac{\partial F'}{\partial z'} (O) \ne 0
    .\]
    
    Этот шаг реализован в файле \textsf{WeierstrassForm.py} в функции
    \textsf{weierstrass\_form\_step2}.

    \item Шаг 3. После второго преобразования, наше уравнение приняло вид:

    Этот шаг реализован в файле \textsf{WeierstrassForm.py} в функции
    \textsf{weierstrass\_form\_step3}.

\end{itemize}


\bigskip
\bigskip

