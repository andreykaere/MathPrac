%!TEX root = ../paper/paper.tex

\begin{center}
    {\normalfont\fontsize{13}{13}\textbf{Приведение кубики к нормальной форме Вейерштрасса}}
\end{center}

Для нашей кубики 
\[
F(x, y, z) = a_{30} x^3 + a_{03} y^3 + a_{00} z^3 + a_{01} y z^2 + a_{10} x z^2 +
a_{11} x y z + a_{21} x^2 y + a_{12} x y^2 + a_{20} x^2 z + a_{02} y^2 z
,\] 
то есть \(a_{ij} \in \Z\) -- коэффициент при мономе \(x^{i} y^{j} z^{3 - i -
j}\). 



\begin{itemize}[leftmargin=0.6cm]
    
    \item Шаг 0. Нахождение рациональных точек перегиба. 

    Для того, чтобы сделать первый шаг в алгоритме приведения неособой
    кубики к нормальной форме Вейерштрасса целочисленным проективным
    преобразованием нам сначала понадобиться найти рациональную точку
    перегиба. Для этого, мы рассмотрим Гессиан нашей кубики \(H(F) := \det \left(
    \frac{\partial F}{\partial x^{i} \partial x^{j}}\right)_{ij}\), где \(x =
    x^{1}, y = x^{2}, z = x^{3}\). Далее, под проективными
    преобразованиями мы будем понимать только целочисленные проективные
    преобразования.

    \begin{theoremf}
        Точки перегиба неособой кубики в \(\C \mathrm{P}^2\) -- в точности
        точки пересечения кубики с её Гессианом.
    \end{theoremf}

    Доказательство можно посмотреть в ~\cite{pra_sol}.

    Итак, для нахождения рациональных точек перегиба, нам достаточно найти
    все общие рациональные корни \(F\) и \(H(F)\). Для этого, мы
    рассмотрим \(F\) и \(H(F)\) как многочлены от \(z\):
    \begin{align*}
        F &= b_0(x, y) z^3 + b_1(x, y) z^2 + b_2(x, y) z + b_3(x, y) \\
        H(F) &= c_0(x, y) z^3 + c_1(x, y) z^2 + c_2(x, y) z + c_3(x, y) 
    ,\end{align*}
    где \(b_k(x, y)\) и \(c_k(x, y)\) -- однородные многочлены от \(x, y\) 
    степени \(k\). 

    Если \(c_0(x, y) \cdot b_0(x, y) = 0\), то есть если
    \(F\) или \(H(F)\) проходит через точку \((0 : 0 : 1)\), то мы можем
    сделать проективное преобразование так, что ни одна из кубик не будет
    через неё проходить. Сделать это можно, например, так: найти точку,
    которая не лежит ни на одной из наших кубик и \textquote{обменять} её
    с \((0 : 0 : 1)\). Итак, будем искать точку в виде \((0 : i : 1), \; i
    = 1, 2, \ldots\).
    Так как кубика неособая, то она не распадающаяся (то есть не содержит
    прямой) и различные кубики могут пересекаться не более, чем в \(9\) 
    точках, то процесс перебора закончится не позднее, чем через \(10\) 
    шагов. (Условие \(c_0 b_0 \ne 0\) нужно, чтобы ... ?)

    Далее считаем, что \(c_0(x, y) \cdot b_0(x, y) \ne 0\). Рассмотрим
    \(R(F, H(F))\) -- результант \(F\) и \(H(F)\) по переменной \(z\) (При
    фиксированных \(x, y\) -- это обычный результант двух многочленов).
    Непосредственной проверкой проверяется, что \(R(F, H(F))\) --
    либо однородный многочлен от \(x, y\) степени \(9\), либо
    тождественный ноль. Перебирая все рациональные корни этого многочлена
    -- получаем все потенциальные рациональные точки перегиба. Подставляя
    их в исходное уравнение \(F\), смотрим, есть ли точка с такими
    рациональными координатами по \(x, y\) с рациональной координатой и по
    \(z\). Таким образом, мы находим все рациональные точки перегиба на
    нашей кубике. 
    
    В программе используется немного модифицированный алгоритм. А именно,
    вместо того, чтобы проверять, является кубика особой, мы сначала находим
    все потенциальные точки перегиба -- это точки пересечения с Гессианом. Это
    реализовано в файле \textsf{InflectionPoints.py} в функции
    \textsf{find\_inflection\_points} с помощью вспомогательных функций
    \textsf{get\_hessian} (которая по кубике выдаёт её Гессиан) и
    \textsf{intersection\_points} (которая выдаёт точки пересечения двух
    произвольных кубик). Далее в функции
    \textsf{find\_non\_singular\_inflection\_point} мы отбираем только
    неособые точки перегиба.

    \item Шаг 1. Выберем какую-то рациональную точку перегиба \(P\) и
    переведём её в точку \((0 : 1 : 0)\) некоторым проективным
    преобразованием. После проективного преобразования, наша кубика имеет
    уравнение \(\widetilde{F}(\widetilde{x}, \widetilde{y}, \widetilde{z})\),
    в котором нет \(\widetilde{y}\). 

    Этот шаг реализован в файле \textsf{WeierstrassForm.py} в функции
    \textsf{weierstrass\_form\_step1}.


    \item Шаг 2. Теперь мы хотим сделать преобразование,
    сохраняющее точку \(O = (0 : 1 : 0)\), так, чтобы
    прямая \(z = 0\) была касательной к кубике \(\widetilde{F}\) в точке
    \(O\), то есть сделать преобразование \((\widetilde{x} :
    \widetilde{y} : \widetilde{z}) \to (x' : y' : z')\), такое что
    для \(F'(x', y', z')\) будет выполнено: 
     \[
    \frac{\partial F'}{\partial x'} (O) = \frac{\partial F'}{\partial y'}
    (O) = 0, \quad \frac{\partial F'}{\partial z'} (O) \ne 0
    .\]
    То есть нет монома \(x y^2\), однако, коэффициент при \(y^2 z\) отличен от
    нуля, и поэтому, разделив на него, можем считать, что он равен \(1\).
    Кроме того, так как \(O\) -- точка перегиба, то касание прямой \(z = 0\) с
    нашей кубикой имеет порядок 3, таким образом нет монома \(x^2 y\). 
    
    Этот шаг реализован в файле \textsf{WeierstrassForm.py} в функции
    \textsf{weierstrass\_form\_step2}.

    \item Шаг 3. После второго преобразования, уравнение нашей кубики стало \(F'(x',
    y', z') = 0\), где 
    \[
    F'(x', y', z') = a_{30}' \left( x' \right)^3 + a_{20}' \left( x' \right)^2
    z' + a_{11}' x' y' z' + \left( y' \right)^2 z' + a_{10}' x' \left(
    z' \right)^2 + a_{01}' y' \left( z' \right)^2 + a_{00}' \left( z'
    \right)^3
    .\] 
    Теперь линейной заменой \(y'' =  y' + (a_{01}' z' + a_{11}' x')/2\) выделяем полный
    квадрат по \(y'\) и уравнение кубики будет:
    \[
    \left( y'' \right)^2 z' = a_{30}'' \left( x' \right)^3 + a_{20}'' \left( x' \right)^2
    z' + a_{10}'' x' \left( z' \right)^2 + a_{00}'' \left( z' \right)^3
    .\] 
    Теперь линейной заменой \(x'' = x' - \dfrac{a_{20}''}{3 a_{30}''} z'\)
    избавляемся от монома \(\left( x' \right)^2 z'\) и получаем:
    \[
    \left( y'' \right)^2 z' = a_{30}''' \left( x'' \right)^3 + a_{10}''' x''
    \left( z' \right)^2 + a_{00}''' \left( z' \right)^3
    .\] 
    Теперь делаем замену \(z'' = a_{30}''' z'\) и получаем:
    \[
    \left( y'' \right)^2 z'' = \left( x'' \right)^3 + a_{30}''' a_{10}''' x''
    \left( z'' \right)^2 + a_{00}''' \left( a_{30}''' \right)^2 \left( z'' \right)^3
    .\] 
    Полагая теперь \(a := a_{30}''' a_{10}''', b := a_{00}''' \left( a_{30}'''
    \right)^2\), получаем требуемую форму: 
    \[
    \left( y'' \right)^2 z'' = \left( x'' \right)^3 + a \, x''
    \left( z'' \right)^2 + b \left( z'' \right)^3
    .\] 

    \begin{remark*}
        Отметим, что все коэффициенты в ходе алгоритма получались
        целочисленными или рациональными (так как совершались только
        целочисленные проективные преобразования). Соответственно, если
        коэффициенты \(a, b\) оказались рациональными, мы их можем сделать
        целыми с помощью замены \(x''' = \frac{1}{d} x'', z''' = \frac{1}{d^3}
        z''\), где \(d\) -- наименьшее общее кратное знаменателей \(a, b\).
    \end{remark*}

    Этот шаг реализован в файле \textsf{WeierstrassForm.py} в функции
    \textsf{weierstrass\_form\_step3}.
\end{itemize}


\bigskip
\bigskip
