\title{Необычная задача кубического представления}  
\date{}  
\author{}

%!TEX root = ../paper/paper.tex

\maketitle  
\section*{Вступление}

Данная статья посвящена поиску решения следующего уравнения
\[
\frac{a} {b + c}  + \frac{b}{a + c} + \frac{c}{a + b} = N, \qquad a, b, c \in
\mathbb{N}
\] 
С первого взгляда может показаться, что найти хоть какое-то решение для малых
\(N\), скажем для \(N = 4\), не составит труда. Однако это не так, ведь самое
короткое решение содержит порядка 80 цифр. Понятно, что прямым перебором такое
решение за разумное время получить нельзя, так как компьютеры способны
выполнять только порядка \(10^{10}\) операций в секунду. Поэтому будут применены 
другие методы. Для начала поймём, что решение исходного уравнения равносильно
нахождению натуральных точек на следующей кубике:
\[
    a^3 + b^3 + c^3 + (1 - N) (a^2 b + b^2 a + a^2 c + c^2 a + b^2 c + c^2 b)
    + (3 - 2 N) a b c = 0
.\] 
Но перед тем как перейти к решению исходной задачи, рассмотрим более простую
задачу -- поиск натуральных точек на квадрике 

\section*{Случай квадрики}
Прежде чем переходить к описанию и нахождения рациональных точек на кубике,
решим данную задачу для квадрик:
\begin{align}
    a x^2 + b y^2 + c z^2 + d x y + e y z + f x z = 0,  \qquad a, b, c, d, e, f \in
    \mathbb{Z} 
.\end{align}
Понятно, что имея одну рациональную точку, методом секущих можно получить
все. Отдельно можно задаться вопросом их явного описания, однако мы это
рассматривать не будем. Для начала упростим задачу -- приведём квадрику
методом Лежандра к одному из следующих видов:
\begin{align*}
    x^2 - A z^2 &= 0 \\
    x^2 - A y z &= 0 \\
    A x^2 + B y^2 - C z^2 &= 0 
\end{align*}
Итак, нас интересуют только целочисленные решения этих уравнений. Случаи 1 и 2
рассматриваются просто (тут надо что-то написать). Случай 3 требует применения
критерия Лежандра. Но для начала стоит понять, что числа $ A, B, C$ можно
считать свободными от квадратов и более того попарно взаимно простыми. Тогда
применим критерий Лежандра, согласно нему уравнение имеет целочисленное
решение тогда и только тогда когда \(BC\) квадратичный вычет по модулю \(A\),
\(AC\) - по модулю \(B\) и \(-AB\)  по модулю \(C\). Эти условия просто
проверить. Однако, следует заметить, что просто знания того, что решение
существует может быть недостаточно (как например в случае основной задачи),
чтобы его можно было найти перебором за разумное время. Однако, на помощь
приходит критерий Хольцера, который утверждает, что если решение существует,
то оно найдётся в следующих пределах:
\[
x_0^2 \leqslant BC, \quad y_0^2 \leqslant AC, \quad z_0^2 \leqslant AB
.\] 

Теперь дадим краткое описание программы.  Программа получает на вход кубику в
форме (1), заданной матрицей \(3 \times 3\). После, действует согласно алгоритму
описанному выше: то есть приводит квадрику к упрощённой форме, применяет
критерий Лежандра, и осуществляет поиск решения в заданных пределах. В итоге
получает рациональную точку лежащую на квадрике, если она существует. Таким
образом случай квадрики полностью разобран.

\bigskip

