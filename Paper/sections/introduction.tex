\title{Необычная задача кубического представления}  
\date{}  
\author{}

\maketitle  
\section*{Вступительное слово}
Эта работа посвящена  нахождению в явном виде решений следующего уравнения: \newline
\[
\begin{aligned}
\frac{a} {b + c}  + \frac{b}{a  + c} + \frac{c}{a+b} = N
\end{aligned} \qquad a, b, c \in \mathbb{N}
\] 
\newline
С первого взгляда может показаться, что найти хоть какое-то решение для малых $N$, скажем для $N = 4$, не составит труда. Однако это не так, ведь самое короткое решение содержит порядка 80 цифр. Понятно, что прямым перебором такое решение за разумное время получить нельзя, так как компьютеры способны выполнять только порядка $10^{10}$ операций в секунду. Поэтому будут применяны другие методы.
Для начала поймем, что решение исходного уравнения равносильно нахождению натуральных точек на следующей кубике:
\newline
\[
\begin{aligned}
	a^3 + b^3 + c^3 + (1 - N) (a^2b + b^2a + a^2c + c^2a + b^2c + c^2b) +(3 - 2N)abc = 0
\end{aligned} 
\] 
\newline
Но перед тем как перейти к решению исходной задачи, рассмотрим более простую задачу - поиск натуральных точек на квадрике 

\section*{Случай квадрики}
Этот случай обладает своей спецификой. Рассматривается задача поиска рациональной точки на кривой:
\newline
\[
\begin{aligned}
	ax^2 + by^2 + cz^2 + dxy + eyz + fxz = 0 
\end{aligned} \qquad a, b, c, d, e, f \in \mathbb{Z}    (1)
\]
Понятно, что имения одну рациональную точку, методом секущих можно получить все. Отдельно можно задаться вопросом их явного описания, однако мы это рассматривать не будем. 
Для начала упростим задачу - приведем квадрику методом Лежадра к одному из следующих видов:
\[
\begin{aligned}
	x^2 - Az^2 = 0 \\
	x^2 - Ayz = 0 \\
	Ax^2 + By^2 - Cz^2 = 0 		\\
\end{aligned} 
,\] 
\newline
Итак, нас интересуют только целочисленные решения этих уравнений. Случаи 1 и 2 рассматриваются просто (тут надо что-то написать). Случай 3 требует применения критерия Лежандра. Но для начала стоит понять, что числа $ A, B, C$ можно считать свободными от квадратов и более того попарно взаимно простыми. Тогда применим критерий Лежандра, согласно нему уравнение 3 имеет целочисленное решение тогда и только тогда когда $BC$ квадратичный вычет по модулю $A$, $AC$ - по модулю $B$ и $-AB$  по модулю $C$. Эти условия просто проверить. Однако, следует заметить, что просто знания того, что решение существует может быть недостаточно (как например в случае основной задачи), чтобы его можно было найти перебором за разумное время. Однако, на помощь приходит критерий Хольцера, который утверждает, что если решение существует, то оно найдется в следующих пределах:
\[
\begin{aligned}
	x_{0}^2 \leqslant BC \\
	y_{0}^2 \leqslant AC \\
	z_{0}^2 \leqslant  AB	\\
\end{aligned} 
,\] 
Теперь дадим краткое описание программы.  Программа получает на вход кубику в форме (1), заданной матрицей 3x3. После, действует согласно алгоритму описанному выше: то есть приводит квадрику к упрощенной форме, применяет критерий Лежандра, и осуществляет поиск решения в заданных пределах. В итоге получает рациональную точку лежащую на квадрике, если она существует. Таким образом случай квадрики полностью разобран.
\newline
\newline

