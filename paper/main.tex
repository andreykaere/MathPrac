\documentclass[a4paper,12pt]{article}

\usepackage[a4paper, left=0.6in, right=0.6in, top=0.85in, bottom=0.85in]{geometry}
\usepackage[utf8]{inputenc}
\usepackage{textcomp}
\usepackage{fontspec}
\usepackage[russian]{babel}
%\usepackage{polyglossia}
%\setdefaultlanguage[spelling=modern]{russian}
%\setotherlanguage{english}
\setmainfont{CMU Serif}
\setsansfont{CMU Sans Serif}
\setmonofont{CMU Typewriter Text}

\usepackage{amsmath, amssymb, amsthm, mathtools, thmtools}

\newcommand\N{\ensuremath{\mathbb{N}}}
\newcommand\R{\ensuremath{\mathbb{R}}}
\newcommand\Z{\ensuremath{\mathbb{Z}}}
\renewcommand\O{\ensuremath{\varnothing}}
\newcommand\Q{\ensuremath{\mathbb{Q}}}
\newcommand\C{\ensuremath{\mathbb{C}}}

\theoremstyle{definition}

\newtheorem{lemmaf}{Лемма}

\newtheorem*{theoremf}{Теорема}
\newtheorem*{statementf}{Утверждение}
\newtheorem*{propositionf}{Предложение}
\newtheorem*{problem}{Задача}

\usepackage{mathdots}
\usepackage{enumitem}
\usepackage{mleftright}
\usepackage{environ}

\NewEnviron{pArray}[1]
{
\left(\begin{array}{#1}
\BODY
\end{array}\right)
}


% figure support
\usepackage{import}
\usepackage{xifthen}
\usepackage{pdfpages}
\usepackage{transparent}

\newcommand{\incfig}[2][1]{%
    \def\svgwidth{#1\columnwidth}
    \import{./figures/}{#2.pdf_tex}
}


\begin{document}
    \begin{center}
        \textbf{Приведение кубики к нормальной форме Вейерштрасса}
    \end{center}

    Для нашей кубики 
    \[
    F(x, y, z) = x^3 + y^3 + z^3 + (1 - N) (x^2 y + x^2 z + y^2 x + y^2 z + z^2 x + z^2 y) +
    (3 - 2 N) x y z 
    \] 
    рациональная точка \(P = (1 : -1 : 0)\) является также точкой перегиба
    (проверяется непосредственно подстановкой в Гессиан).

    \begin{itemize}[leftmargin=1cm]
        \item Шаг 1. Переведём точку \(P = (1 : -1 : 0)\) в точку
        \((0 : 1 : 0)\) некоторым проективным преобразованием. Например,
        подойдёт такое:
        \[
        \lambda
        \begin{pmatrix}
            \widetilde{x} \\
            \widetilde{y} \\
            \widetilde{z} \\
        \end{pmatrix} 
        = 
        \begin{pmatrix}
            1 & 1 & 0 \\
            1 & 0 & 0 \\
            0 & 0 & 1 \\
        \end{pmatrix}
        \begin{pmatrix}
            x \\
            y \\
            z \\
        \end{pmatrix}
        \]
        или
        \[
        \begin{pmatrix}
            x \\
            y \\
            z \\
        \end{pmatrix} =
        \lambda
        \begin{pmatrix}
            0 & 1 & 0 \\
            1 & -1 & 0 \\
            0 & 0 & 1 \\
        \end{pmatrix}
        \begin{pmatrix}
            \widetilde{x} \\
            \widetilde{y} \\
            \widetilde{z} \\
        \end{pmatrix} 
        .\] 
        

        Тогда наше уравнение преобразуется в следующее:  
        \begin{align*}
            \widetilde{F}(\widetilde{x}, \widetilde{y}, \widetilde{z}) &= 
         \left(\widetilde{z}\right)^3
         -(N + 2) \left(\widetilde{x}\right)^2 \widetilde{y} + (1 - N)
         \left(\widetilde{x}\right)^2 \widetilde{z} + (N + 2) \,
         \widetilde{x} \left(\widetilde{y}\right)^2 \\ 
         &+ (1 - N) \, \widetilde{x} \left(\widetilde{z}\right)^2 + 
         \left(\widetilde{x}\right)^3 + \widetilde{x} \,
         \widetilde{y} \, \widetilde{z} -
         \left(\widetilde{y}\right)^2 \widetilde{z}
        .\end{align*}
        

        \item Шаг 2. Теперь мы хотим сделать преобразование,
        сохраняющее точку \(O = (0 : 1 : 0)\), так, чтобы
        прямая \(z = 0\) была касательной к кубике \(\widetilde{F}\) в точке
        \(O\), то есть сделать преобразование \((\widetilde{x} :
        \widetilde{y} : \widetilde{z}) \to (x' : y' : z')\), такое что
        для \(F'(x', y', z')\) будет выполнено: 
         \[
        \frac{\partial F'}{\partial x'} (O) = \frac{\partial F'}{\partial y'}
        (O) = 0, \quad \frac{\partial F'}{\partial z'} (O) \ne 0
        .\]
        Пусть имеет место соотношение
        \[
        \begin{pmatrix}
            \widetilde{x} \\
            \widetilde{y} \\
            \widetilde{z} \\
        \end{pmatrix} 
        = 
        \begin{pmatrix}
            c_{11} & c_{12} & c_{13} \\
            c_{21} & c_{22} & c_{23} \\
            c_{31} & c_{32} & c_{33} \\
        \end{pmatrix}
        \begin{pmatrix}
            x' \\
            y' \\
            z' \\
        \end{pmatrix}
        .\] 
        Так как точка \(O\) должна быть неподвижна, то:
        \[
        \begin{pmatrix}
            0 \\
            1 \\
            0 \\
        \end{pmatrix}
        = 
        \begin{pmatrix}
            c_{11} & c_{12} & c_{13} \\
            c_{21} & c_{22} & c_{23} \\
            c_{31} & c_{32} & c_{33} \\
        \end{pmatrix}
        \begin{pmatrix}
            0 \\
            1 \\
            0 \\
        \end{pmatrix}
        ,\] 
        откуда \(c_{12} = 0, c_{22} = 1, c_{32} = 0\). 
            
        Кроме того, поскольку \(F'(x', y', z') = \widetilde{F}(\widetilde{x}(x', y', z'),
        \widetilde{y}(x', y', z'), \widetilde{z}(x', y', z'))\), то мы имеем:
        %
        \begin{align*}
        \frac{\partial F'}{\partial x'} (O) &= \frac{\partial
        \widetilde{F}}{\partial \widetilde{x}} (\widetilde{x}(O),
        \widetilde{y}(O), \widetilde{z}(O))
        \frac{\partial \widetilde{x}}{\partial x'} (O) + \frac{\partial
        \widetilde{F}}{\partial \widetilde{y}} (\widetilde{x}(O),
        \widetilde{y}(O), \widetilde{z}(O))
        \frac{\partial \widetilde{y}}{\partial x'} (O) \\ 
        &+ \frac{\partial
        \widetilde{F}}{\partial \widetilde{z}} (\widetilde{x}(O),
        \widetilde{y}(O), \widetilde{z}(O))
        \frac{\partial \widetilde{z}}{\partial x'} (O) \\ &= 
        \frac{\partial \widetilde{F}}{\partial \widetilde{x}}(c_{12}, c_{22},
        c_{32}) c_{11} 
        + \frac{\partial \widetilde{F}}{\partial
        \widetilde{y}}(c_{12}, c_{22}, c_{32}) c_{21} + \frac{\partial
        \widetilde{F}}{\partial \widetilde{z}}(c_{12}, c_{22}, c_{32}) c_{31} \\
        &= 
        \frac{\partial \widetilde{F}}{\partial \widetilde{x}}(0, 1, 0) c_{11} 
        + \frac{\partial \widetilde{F}}{\partial
        \widetilde{y}}(0, 1, 0) c_{21} + \frac{\partial
        \widetilde{F}}{\partial \widetilde{z}}(0, 1, 0) c_{31} = (2 + N) c_{11}
        - c_{31}
        .\end{align*}
        Аналогично получаем:
        \begin{align*}
        \frac{\partial F'}{\partial y'} (O) &= \left( \frac{\partial
        \widetilde{F}}{\partial \widetilde{x}} 
        \frac{\partial \widetilde{x}}{\partial y'}  + \frac{\partial
        \widetilde{F}}{\partial \widetilde{y}} 
        \frac{\partial \widetilde{y}}{\partial y'}  + \frac{\partial
        \widetilde{F}}{\partial \widetilde{z}} 
        \frac{\partial \widetilde{z}}{\partial y'} \right) \! \! \Bigg|_{O}  = (2 + N) \frac{\partial \widetilde{x}}{\partial y'}(O) - \frac{\partial
        \widetilde{z}}{\partial y'} (O) = (2 + N) c_{12} - c_{32} \\
        \frac{\partial F'}{\partial z'} (O) &= \left( \frac{\partial
        \widetilde{F}}{\partial \widetilde{x}} 
        \frac{\partial \widetilde{x}}{\partial z'}  + \frac{\partial
        \widetilde{F}}{\partial \widetilde{y}} 
        \frac{\partial \widetilde{y}}{\partial z'}  + \frac{\partial
        \widetilde{F}}{\partial \widetilde{z}} 
        \frac{\partial \widetilde{z}}{\partial z'} \right) \! \! \Bigg|_{O}  =
        (2 + N) \frac{\partial \widetilde{x}}{\partial z'}(O) - \frac{\partial
        \widetilde{z}}{\partial z'} (O) = (2 + N) c_{13} - c_{33}
        .\end{align*}
        Итак, мы получаем следующие условия на матрицу \(C = (c_{ij})\): 
        \begin{align*}
            (2 + N) c_{11} - c_{31} &= 0 \\
            (2 + N) c_{12} - c_{32} &= 0 \\
            (2 + N) c_{13} - c_{33} &\ne  0 \\
            c_{12} = c_{32} &= 0 \\
            c_{22} &= 1
        .\end{align*}
        Итого, получаем, что в качестве искомой матрицы можно взять следующую:
        \[
        \begin{pmatrix}
            1 & 0 & 1 \\
            1 & 1 &  1\\
            2 + N & 0  & N + 1 \\
        \end{pmatrix}
        .\] 
        А тогда наше преобразование будет иметь вид:
        \[
        \lambda \begin{pmatrix}
            x' \\
            y' \\
            z' \\
        \end{pmatrix} = 
        \begin{pmatrix}
            -N - 1 & 0 & 1 \\
            -1 & 1 & 0 \\
            N + 2 & 0 & -1 \\
            % N + 3 & 0 & -1 \\
            % -1 & 1 & 0 \\
            % -N - 2 & 0 & 1 \\ 
        \end{pmatrix}
        \begin{pmatrix}
            \widetilde{x} \\
            \widetilde{y} \\
            \widetilde{z} \\
        \end{pmatrix}
        .\] 
        
        \item Шаг 3. После второго преобразования, наше уравнение приняло вид:
        \begin{align*}
            F'(x', y', z') &= \left(2 N^2 + 11 N + 15\right) \left( x' \right)^3
            + \left(5 N^2 + 24 N + 28\right) \left( x' \right)^2 z \\ &+ \left(4 N^2
            + 17 N + 18\right) x \left( z' \right)^2 + \left(N^2 + 4 N +
            4\right) \left( z' \right)^3  + x y z + \left( y' \right)^2 z + y
            \left( z' \right)^2 
        .\end{align*}

        
    \end{itemize}
\end{document}

